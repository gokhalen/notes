\documentclass{article}
\newcommand{\beq}{\begin{equation}}
\newcommand{\eeq}{\end{equation}}
\newcommand{\ber}{\begin{eqnarray}}
\newcommand{\eer}{\end{eqnarray}}
\newcommand{\nn}{\nonumber}
\newcommand{\dd}[1]{\frac{d}{d{#1}} }
\newcommand{\ddeps}{\frac{d}{d\epsilon}\Big{|}_{\rightarrow{0}}}
\newcommand{\pdd}[1]{\frac{\partial}{\partial{#1}}}
\newcommand{\pddd}[2]{\frac{\partial^2}{\partial{#1}\partial{#2}}}
\newcommand{\bD}{{\mathbf{D}}}
\newcommand{\notimplies}{\;\not\!\!\!\implies}
\usepackage{amsmath}
\DeclareMathOperator{\sign}{sign}
\usepackage{amsfonts}
\usepackage{bm}
\usepackage{booktabs}
\usepackage[hyphens]{url}
\usepackage{amssymb} 
\usepackage[utf8]{inputenc} 
%\usepackage[ngerman]{babel} 
\usepackage[T1]{fontenc}
\usepackage[margin=2.5cm]{geometry}
\usepackage{listings}
\usepackage{multirow}
\usepackage{tikz}
\usepackage{cancel}
\definecolor {processblue}{cmyk}{0.96,0,0,0}
\usetikzlibrary {positioning}
\usepackage{hyperref}
% for booktabs
\newcommand{\ra}[1]{\renewcommand{\arraystretch}{#1}}
\begin{document}
\title{Adjoint method explained}
\author{Nachiket Gokhale}
\date{\today}
\maketitle
%
%
%
\section{Formulation}
The presentation of the adjoint method in Oberai,Gokhale,Feijoo 2003 is not clear. It is asserted that on surfaces satisfying $a(\pmb{w},\pmb{u};\mu)=(\pmb{w},\pmb{h})_{\Gamma_h}$, $\delta{L}=\delta{\pi}$ and hence $\delta{L}_{\mu}=\delta{\pi}_{\mu}$. Here, we provide an algebraic explanation of this.
%
%
%
\section{Straightforward calculation}
%
%
\beq
\label{eqn:pi}
\pi(\pmb{u}(\mu)) = \frac{1}{2}\|\pmb{u}-\pmb{u}^m\|^2
\eeq
%
Where $\pmb{u}$ obeys the elasticity equations $a(\pmb{w},\pmb{u};\mu)=(\pmb{w},\pmb{h})_{\Gamma_h}$. Differentiating (\ref{eqn:pi}) with respect to $\mu$ we get
%
\beq
\label{eqn:deltapi}
\delta{\pi} = (\pmb{u}-\pmb{u}^m,\delta{\pmb{u}})
\eeq
%
Where $\delta{\pmb{u}}$ satisfies
%
\beq
\label{eqn:varumu}
a(\pmb{w},\delta\pmb{u};\mu) + a(\pmb{w},\pmb{u};\delta\mu) = 0 ,\,\forall\,\pmb{w}
\eeq
%
which is nothing but the elasticity equation differentiated wrt $\mu$. To find the change in $\pi$ due to a change $\delta\mu$ we calculated first $\delta{\pmb{u}}$ using equation (\ref{eqn:varumu}) and then use it in equation (\ref{eqn:deltapi})
%
%
%
\section{Adjoint calculation}
Consider the following function
%
%
%
\beq
L(\pmb{u},\pmb{w},\mu) = \frac{1}{2}\|\pmb{u}-\pmb{u}^m\|^2 + a(\pmb{w},\pmb{u};\mu)-(\pmb{w},\pmb{h})_{\Gamma_h}
\eeq
where $\pmb{u},\pmb{w},\mu$ are free to be whatever they want. Taking variations we get
%
\beq
\delta{L} = \delta{L}_{\pmb{u}} + \delta{L}_{\pmb{w}} + \delta{L}_{\mu}
\eeq
%
If we choose $\pmb{w}$ such that $\delta{L}_{\pmb{u}}=0,\,\forall\,\delta\pmb{u}$, and $\pmb{u}$ such that $\delta{L}_{\pmb{w}}=0,\,\forall\,\delta\pmb{w}$, we get
%
\beq
\label{eqn:deltaLmu}
\delta{L} = \delta{L}_{\mu} = a(\pmb{w},\pmb{u};\delta\mu)
\eeq
%
\textbf{CLAIM}: $\delta{L}_{\mu}$ from (\ref{eqn:deltaLmu}) is the same as $\delta{\pi}$ from (\ref{eqn:deltapi})
\beq
\delta{L}_{\mu} = a(\pmb{w},\pmb{u};\delta\mu) = \delta{\pi}
\eeq
%
\textbf{PROOF}:
Note: $\delta{L}_{\pmb{w}}=0,\,\forall\,\delta\pmb{w}$ implies equation (\ref{eqn:varumu})
\ber
\delta{L}_{\mu} = a(\pmb{w},\pmb{u};\delta\mu) &=& -a(\pmb{w},\delta\pmb{u};\mu) \qquad \text{ using } (\ref{eqn:varumu}) \\
\label{eqn:Ldeltamu2}
&=& -a(\delta\pmb{u},\pmb{w};\mu) \qquad \text{using symmetry of } a(\cdot,\cdot,\mu) 
\eer
$\delta{L}_{\pmb{u}}=0,\,\forall\,\delta\pmb{u}$ reduces to 
\beq
(\pmb{u}-\pmb{u}^m,\delta{\pmb{u}}) + a(\delta\pmb{u},\pmb{w},\mu) = 0
\eeq
Using the above eqation in (\ref{eqn:Ldeltamu2}) we get
\beq
\delta{L}_{\mu} = (\pmb{u}-\pmb{u}^m,\delta{\pmb{u}})
\eeq
which is the same as equation (\ref{eqn:pi}). QED.
%The condition $\delta{L}_{\pmb{w}}=0,\,\forall\,\delta\pmb{w}$ implies
%\beq
%a(\delta\pmb{w},\pmb{u};\mu)-(\delta\pmb{w},\pmb{h})_{\Gamma_h} = 0\,\forall\,\delta\pmb{w}
%\eeq
%Which can be differentied wrt $\mu$ to get
%
%\beq
%a(\delta\pmb{w},\pmb{u};\delta\mu) + a(\delta\pmb{w},\delta\pmb{u};\mu) = 0\,\forall\,\delta\pmb{w}
%\eeq
%
\end{document}

