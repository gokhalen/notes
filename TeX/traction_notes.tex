\documentclass{article}
\newcommand{\beq}{\begin{equation}}
\newcommand{\eeq}{\end{equation}}
\newcommand{\ber}{\begin{eqnarray}}
\newcommand{\eer}{\end{eqnarray}}
\usepackage{amsmath}
\usepackage{amsfonts}
\usepackage{url}
\begin{document}
\title{Traction notes}
\author{Nachiket Gokhale}
\date{\today}
\maketitle
\section{Traction notes}
\subsection{My misconception}
Let us assume that the boundary can be divided as follows. $\Gamma=\Gamma_h\cup\Gamma_q$ and that all three displacement degrees of freedom are specified on $\Gamma_q$ and all three traction components are specified on $\Gamma_h$. This makes the notation simpler. The traction contribution to the rhs vector resulting from $(\mathbf{w},\mathbf{h})_{\Gamma_h}$
\beq
(N_A\mathbf{e}_{i},\mathbf{h})_{\Gamma_h}
\eeq
My mistake was that it seemed that we were assembling the vector $\mathbf{h}$ into the location $(A,i)$. That is not so, due to the defition of the innerproduct $(\cdot,\cdot)_{\Gamma_h}$ which maps a pair of vectors to a real number. The simplification is
\beq
(N_A\mathbf{e}_{i},\mathbf{h})_{\Gamma_h} = \int_{\Gamma_h} N_{A}\mathbf{e}_{i}\cdot\mathbf{h}\,d\Gamma_{h} = \int_{\Gamma_h}N_Ah_id\Gamma_{h}
\eeq
Note $\mathbf{e}_{i}$ is a vector with zeros everywhere except at its $i^{th}$ component, which is $1$.
\subsection{Traction elements not aligned with the co-ordinate axes}
If traction elements are not aligned with the coordinate axes, performing integration over them is non-trivial. Some references are:\\
\url{http://sites.science.oregonstate.edu/math/home/programs/undergrad/CalculusQuestStudyGuides/vcalc/surfint/surfint.html}\\
\url{https://www.khanacademy.org/math/multivariable-calculus/integrating-multivariable-functions/surface-integrals-articles/a/surface-integrals}\\
\url{https://www.khanacademy.org/math/multivariable-calculus/integrating-multivariable-functions/surface-integrals-articles/a/surface-integral-example}\\
\url{https://www.khanacademy.org/math/multivariable-calculus/integrating-multivariable-functions/line-integrals-for-scalar-functions-articles/a/line-integrals-in-a-scalar-field}
\subsection{Integrating over traction elements}
In 2D traction elements are lines. See in Hughes' linear book, Integration over a line in 2D: Exercise 8 on pg 161. Exercise 10,11,12,13 on pg 163.

Integrating over a boundary of a 3D trilinear element: we want to integrate over an arbitrary plane (which is not parallel to XY, XZ, or YZ planes) in 3D, we can regard x,y as independent variables and treat z=f(x,y). f depends on the equation of the plane.  Find the angle theta between the plane and the XY plane, by finding the angle between their normals. Change a surface element of the plane $dS=\frac{dxdy}{cos(\theta)}$ or something similar. Now this problem reduces to a double integration problem. So, if the nodes of the 2D traction element living ing 3D space we're integrating over are: $(x_a,y_a,z_a), a=1,2,3,4$. So,
\beq
\int_{S}g(x,y,z)\,dS = \int_{\Omega} g(x,y,f(x,y))\,\frac{dxdy}{cos(\theta)}
\eeq
where, $\Omega$ is the region defined by projecting the plane we're intgrating over to the XY plane. i.e. the quadrilateral with the nodes $(x_a,y_a,0), a=1,2,3,4$. This procedure will not work when the plane we're integrating over is perpendicular to the XY-plane. $cos(\theta)$ will be $0$. Those cases must be handled as special cases.
\subsection{Intgration over curved surfaces}
If we're integrating over non-planar surfaces, $cos(\theta)$ will not be a constant, but will vary at every point on the surface. By dividing the non-planar surface into many small planar surfaces, we can evaluate surface integrals. If the small surfaces become perpendicular to the $XY$ plane then those cases must be handled separately. Dividing a surface into planes and calculating the integral is similar to to 1D numerical integration, in which we approximate the function to be intgrated by straigt lines over a small segments, calculate the area under each straight line and then sum it up.
\subsection{Order of integration}
The boundary traction integral is given by:
\beq
\label{eqn:boundaryintegral}
(\mathbf{w},\mathbf{h})_{\Gamma_h} = \int_{\Gamma_h}w_ih_i d\Gamma_h
\eeq
When we're implementing the above in 2D, the question arises: Should be integrate from node 1 to node 2 or from node 2 to node 1?
Based on
\beq
\int_{x_1}^{x_2} dx = -\int_{x_2}^{x_1} dx
\eeq
we might think that order matters. However, this is not correct, because eqn \ref{eqn:boundaryintegral} is a line integral. As explained here \url{https://math.stackexchange.com/questions/3803685/sign-of-a-line-integral/3803697} the direction of integration does not matter for a scalar line integral. Line integrals are different creatures from ordinary integrals. Also see \url{https://mathinsight.org/line_integral_independent_parametrization}
\end{document}
