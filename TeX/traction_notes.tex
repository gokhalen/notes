\documentclass{article}
\newcommand{\beq}{\begin{equation}}
\newcommand{\eeq}{\end{equation}}
\newcommand{\ber}{\begin{eqnarray}}
\newcommand{\eer}{\end{eqnarray}}
\newcommand{\pdd}[2]{\frac{\partial{#1}}{\partial{#2}}}
\newcommand{\nn}{\nonumber}
\usepackage{amsmath}
\usepackage{amsfonts}
\usepackage{url}
\begin{document}
\title{Traction notes}
\author{Nachiket Gokhale}
\date{\today}
\maketitle
\section{Traction notes}
\subsection{My misconception}
Let us assume that the boundary can be divided as follows. $\Gamma=\Gamma_h\cup\Gamma_q$ and that all three displacement degrees of freedom are specified on $\Gamma_q$ and all three traction components are specified on $\Gamma_h$. This makes the notation simpler. The traction contribution to the rhs vector resulting from $(\mathbf{w},\mathbf{h})_{\Gamma_h}$
\beq
(N_A\mathbf{e}_{i},\mathbf{h})_{\Gamma_h}
\eeq
My mistake was that it seemed that we were assembling the vector $\mathbf{h}$ into the location $(A,i)$. That is not so, due to the defition of the innerproduct $(\cdot,\cdot)_{\Gamma_h}$ which maps a pair of vectors to a real number. The simplification is
\beq
(N_A\mathbf{e}_{i},\mathbf{h})_{\Gamma_h} = \int_{\Gamma_h} N_{A}\mathbf{e}_{i}\cdot\mathbf{h}\,d\Gamma_{h} = \int_{\Gamma_h}N_Ah_id\Gamma_{h}
\eeq
Note $\mathbf{e}_{i}$ is a vector with zeros everywhere except at its $i^{th}$ component, which is $1$.
\subsection{Traction elements not aligned with the co-ordinate axes}
If traction elements are not aligned with the coordinate axes, performing integration over them is non-trivial. Some references are:\\
\url{http://sites.science.oregonstate.edu/math/home/programs/undergrad/CalculusQuestStudyGuides/vcalc/surfint/surfint.html}\\
\url{https://www.khanacademy.org/math/multivariable-calculus/integrating-multivariable-functions/surface-integrals-articles/a/surface-integrals}\\
\url{https://www.khanacademy.org/math/multivariable-calculus/integrating-multivariable-functions/surface-integrals-articles/a/surface-integral-example}\\
\url{https://www.khanacademy.org/math/multivariable-calculus/integrating-multivariable-functions/line-integrals-for-scalar-functions-articles/a/line-integrals-in-a-scalar-field}
\subsection{Integrating over traction elements}
In 2D traction elements are lines. See in Hughes' linear book, Integration over a line in 2D: Exercise 8 on pg 161. Exercise 10,11,12,13 on pg 163.

Integrating over a boundary of a 3D trilinear element: we want to integrate over an arbitrary plane (which is not parallel to XY, XZ, or YZ planes) in 3D, we can regard x,y as independent variables and treat z=f(x,y). f depends on the equation of the plane.  Find the angle theta between the plane and the XY plane, by finding the angle between their normals. Change a surface element of the plane $dS=\frac{dxdy}{cos(\theta)}$ or something similar. Now this problem reduces to a double integration problem. So, if the nodes of the 2D traction element living ing 3D space we're integrating over are: $(x_a,y_a,z_a), a=1,2,3,4$. So,
\beq
\int_{S}g(x,y,z)\,dS = \int_{\Omega} g(x,y,f(x,y))\,\frac{dxdy}{cos(\theta)}
\eeq
where, $\Omega$ is the region defined by projecting the plane we're intgrating over to the XY plane. i.e. the quadrilateral with the nodes $(x_a,y_a,0), a=1,2,3,4$. This procedure will not work when the plane we're integrating over is perpendicular to the XY-plane. $cos(\theta)$ will be $0$. Those cases must be handled as special cases.
{\textbf{Important}} In general, the face of a 3D trilinear hex element is not plane. See Section (\ref{sect:3dhexcorrect}) for the correct implementation. 

\subsection{Intgration over curved surfaces}
If we're integrating over non-planar surfaces, $cos(\theta)$ will not be a constant, but will vary at every point on the surface. By dividing the non-planar surface into many small planar surfaces, we can evaluate surface integrals. If the small surfaces become perpendicular to the $XY$ plane then those cases must be handled separately. Dividing a surface into planes and calculating the integral is similar to to 1D numerical integration, in which we approximate the function to be intgrated by straigt lines over a small segments, calculate the area under each straight line and then sum it up.
\subsection{Order of integration}
The boundary traction integral is given by:
\beq
\label{eqn:boundaryintegral}
(\mathbf{w},\mathbf{h})_{\Gamma_h} = \int_{\Gamma_h}w_ih_i d\Gamma_h
\eeq
When we're implementing the above in 2D, the question arises: Should be integrate from node 1 to node 2 or from node 2 to node 1?
Based on
\beq
\int_{x_1}^{x_2} dx = -\int_{x_2}^{x_1} dx
\eeq
we might think that order matters. However, this is not correct, because eqn \ref{eqn:boundaryintegral} is a line integral. As explained here \url{https://math.stackexchange.com/questions/3803685/sign-of-a-line-integral/3803697} the direction of integration does not matter for a scalar line integral. Line integrals are different creatures from ordinary integrals. Also see \url{https://mathinsight.org/line_integral_independent_parametrization}
\section{\label{sect:3dhexcorrect}Integrating traction correctly for 3D hex elements}
The methods in previous sections are correct for planes and curved surfaces, but they are not the way they are implemented in finite element codes. The correct way is from \url{https://classes.engineering.wustl.edu/mase5510/} Chapter 5, Section 5.6.2. This is summarized below. The mapping functions
\beq
x = Q_x(\xi,\eta,\zeta) \qquad  y = Q_y(\xi,\eta,\zeta) \qquad   z = Q_z(\xi,\eta,\zeta) 
\eeq
map the parent element to the physical/global element. Explicitly,
\beq
Q_x = \sum_{A=1}^{A=8}N_A(\xi,\eta,\zeta)x_A
\eeq
and similar mapping formulae for the others.\\
Given the mapping functions, each face of the hex is parameterized by imposing the appropriate restriction on the mapping function. For example, if integration is to be performed on the face of a hexahedron that corresponds to $\zeta$ = 1 then, the parametric form of the surface becomes:
\beq
x = Q_x(\xi,\eta,1) \qquad  y = Q_y(\xi,\eta,1) \qquad   z = Q_z(\xi,\eta,1) 
\eeq
Then,
\beq
\int_{{d\Omega}_{\zeta=1}} F(x,y,z)dS = \int_{-1}^{1}\int_{-1}^{1}F(x(\xi,\eta,1),y(\xi,\eta,1),z(\xi,\eta,1))\Big{|}\pdd{\stackrel{\rightarrow}{r}}{\xi}\times\pdd{\stackrel{\rightarrow}{r}}{\eta}\Big{|}{d\xi}d\eta
\eeq
where,
\beq
\stackrel{\rightarrow}{r} = x(\xi,\eta,\zeta)\stackrel{\rightarrow}{e_x} + y(\xi,\eta,\zeta)\stackrel{\rightarrow}{e_y} + z(\xi,\eta,\zeta)\stackrel{\rightarrow}{e_z}
\eeq
So, to make things more explicit,
\beq
T=\int_{{d\Omega}_{\zeta=1}} N_A(x,y,z)h_i dS = \int_{{d\Omega}_{\zeta=1}} N_A(x,y,z)(\sum_{B=1}^{B=8}N_{B}(x,y,z)h_{iB} )dS
\eeq
4 out of 8 shape functions will vanish in $\sum_{B=1}^{B=8}N_{B}(x,y,z)h_{iB}$ because that summation is over a face.
\beq
T =  \int_{-1}^{1}\int_{-1}^{1}  n_A(\xi,\eta,\zeta) (\sum_{B=1}^{B=8}n_{B}(\xi,\eta,\zeta)h_{iB})       \Big{|}\pdd{\stackrel{\rightarrow}{r}}{\xi}\times\pdd{\stackrel{\rightarrow}{r}}{\eta}\Big{|}{d\xi}d\eta
\eeq
Where,
\beq
n_A(\xi,\eta,\zeta) = N_A(x(\xi,\eta,\zeta),y(\xi,\eta,\zeta),z(\xi,\eta,\zeta))
\eeq
Shape functions are defined in the parent domain, as noted in the shape\_functions.tex document.
\end{document}
