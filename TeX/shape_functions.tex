\documentclass{article}
\newcommand{\beq}{\begin{equation}}
\newcommand{\eeq}{\end{equation}}
\newcommand{\ber}{\begin{eqnarray}}
\newcommand{\eer}{\end{eqnarray}}
\usepackage{amsmath}
\begin{document}
\title{Shape functions}
\author{Nachiket Gokhale}
\date{\today}
\maketitle
\section{Shape functions}
Hughes' linear book does not define the transformation from the global domain to the local domain in an easy to understand manner. The notation is confusing esp. on equation (1.15.5) where $N_a(x(\xi))$ becomes $N_a(\xi)$. \\
Let the element in global coordinates be $[x_1,x_2]$. Let $N_1(x)$ and $N_2(x)$ be the shape functions associated with nodes $1$ and $2$. \\
In finite elements we are often interested in evaluating integrals of the form
\begin{equation}
  \label{eqn:mainintegral}
  I=\int_{x_1}^{x_2}f(x)dx
\end{equation}
Most of the time we replace $f(x)$ by its restriction to the finite dimensional space under consideration
\begin{equation}
  \label{eqn:restriction}
f^h(x)=\sum_{A=1}^{A=2}N_{A}(x)f_{A},\text{ where } f_A=f(x_A)
\end{equation}
Equation (\ref{eqn:mainintegral}) is typically evaluated by transforming the coordinates from the global domain $[x_1,x_2]$ to the local domain $[-1,1]$ using a transformation $\xi:[x_1,x_2]\rightarrow[-1,1]$. One convenient transformation which accomplishes this is
\beq
\label{eqn:globaltolocal}
\xi(x) = \frac{2x -x_1 - x_2}{x_2-x_1}
\eeq
The inverse transformation $x:[-1,1]\rightarrow[x_1,x_2]$ is
\beq
\label{eqn:localtoglobal}
x(\xi) = \frac{(x_2 - x_1)\xi + x_1 + x_2}{2}
\eeq
So, $x$ is regarded as a function of the local coordinates $\xi$, which are regarded as independent variables. Therefore, using (\ref{eqn:restriction}) in (\ref{eqn:mainintegral})
\ber
I &=& \int_{\xi=-1}^{\xi=+1}\sum_{A=1}^{A=2}\,N_{A}(x(\xi))f_{A}\,\, x(\xi)_{,\xi} d\xi \text{ where } x(\xi)_{,\xi}=\frac{\partial x(\xi)}{\partial \xi} \nonumber \\
&=&  \sum_{A=1}^{A=2}\int_{\xi=-1}^{\xi=+1}\,N_{A}(x(\xi))f_{A}\,\, x(\xi)_{,\xi} d\xi \nonumber \\
&=&  \sum_{A=1}^{A=2}\int_{\xi=-1}^{\xi=+1}\,n_{A}(\xi)f_{A}\,\, x(\xi)_{,\xi} d\xi \text{ where } n_A(\xi)=N_A(x(\xi)) \label{eqn:localintegral}
\eer
The sum $n_A(\xi)f_A$ can be thought of as a local interpolation of $f$. We can define a local interpolant of $f$.
\beq
\label{eqn:localinterpolantf}
\hat{f}(\xi) = n_A(\xi)f_A
\eeq
If we can find the form of $n_A(\xi)$, we can evaulate the integral (\ref{eqn:localintegral}) numerically. To do this, take the definition of the function $N_1(x)$ in global coordinates, and put (\ref{eqn:localtoglobal}) for $x$. Let us take
\beq
\label{eqn:defglobalN1}
N_1(x) = \frac{x_2 - x}{x_2 - x_1}.
\eeq
Therefore,
\ber
n_1(\xi)&=&N_1(x(\xi)) = N_1\Big(\frac{(x_2 - x_1)\xi + x_1 + x_2}{2}\Big) \nonumber \\
        &=& \frac{x_2 - \frac{(x_2-x_1)\xi + x_1 + x_2}{2}}{x_2-x_1} \nonumber \\
        &=& \frac{\frac{1}{2}\Big[2x_2-[(x_2-x_1)\xi + x_1 + x_2]\Big]}{x_2-x_1} \nonumber \\
        &=& \frac{\frac{1}{2}\Big[2x_2-(x_2-x_1)\xi - x_1 - x_2\Big]}{x_2-x_1} \nonumber \\
        &=& \frac{x_2 - x_1 - (x_2-x_1)\xi}{2(x_2 - x_1)} \nonumber \\
        &=& \frac{(x_2-x_1)(1-\xi)}{2(x_2 - x_1)} \nonumber \\
        &=& \frac{1}{2}(1-\xi) \label{eqn:localbasis1}
\eer
Similarly $n_2$ can be derived to be
\beq
\label{eqn:localbasis2}
n_2(\xi) = \frac{1}{2}(1+\xi)
\eeq
\subsection{Isoparameteric finite elements}
The choice made for the global finite element basis in equation (\ref{eqn:defglobalN1}) and for the transformation from global to local coordinates in equation (\ref{eqn:globaltolocal}) leads to the definition of $n_1$ and $n_2$ in equations (\ref{eqn:localbasis1},\ref{eqn:localbasis2}). This particular basis has the property that in addition to being the basis for the local interpolation in equation (\ref{eqn:localinterpolantf}) the we can also express (\ref{eqn:localtoglobal}) as follows:
\ber
x(\xi) &=& \frac{(x_2 - x_1)\xi + x_1 + x_2}{2} \nonumber \\
       &=& \frac{1}{2}\Big[(x_1 - x_1\xi) + (x_2 + x_2\xi)\Big] \nonumber \\
       &=& \frac{1}{2}(1-\xi)x_1 + \frac{1}{2}(1+\xi)x_2 \nonumber \\
       &=& n_1(\xi)x_1 + n_2(\xi)x_2 \label{eqn:localtoglobalusingbasis}
\eer
The property of being able to express the local interpolant (\ref{eqn:localinterpolantf}) and the the local to global transformation (\ref{eqn:localtoglobalusingbasis}) using the same basis is called 'isoparametric'. Such elements are referred to as isoparametric finite elements.
\subsection{Global $N(\mathbf{x})$ not defined \& Completeness}
Hughes does not explicitly define $N(\mathbf(x))$ in global coordinates as we have done in equation (\ref{eqn:defglobalN1}). Hughes defines completenes in the global and element domain. But because we do not have explicit definitions of $N(\mathbf(x))$ in the global domain we cannot verify this for ourselves directly. Hughes does it in a different way on pg 117 where he proves $u^h(\xi,\eta)=u^h(\mathbf{x})$. Notation as usual is confusing.
\end{document}
