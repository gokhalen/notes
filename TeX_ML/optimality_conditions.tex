\documentclass{article}
\newcommand{\beq}{\begin{equation}}
\newcommand{\eeq}{\end{equation}}
\newcommand{\ber}{\begin{eqnarray}}
\newcommand{\eer}{\end{eqnarray}}
\newcommand{\nn}{\nonumber}
\usepackage{graphicx}
\usepackage{amsmath}
\usepackage{amsfonts}
\usepackage{url}
\begin{document}
\title{Optimality conditions}
\author{Nachiket Gokhale}
\date{\today}
\maketitle
\section{Introduction}
A (slightly hand waving) proof of first and second order optimality conditions is given. Let $x_0$ be a local minimum of $f(x)$ i.e. $f(x_0) \leq f(x_0+h) \,\forall\, h $ in a small neighborhood of $x_0$. The $\leq$ is important. Changing it to $<$ does not work.
\section{Proof: first condition}
\ber
f(x_0) &\leq& f(x_0+h)  \,\forall\, h \text{ in a small neighborhood around } x \\
f(x_0) &\leq& f(x_0) + f'(x_0)h + ...\\ 
f(x_0) &\leq& f(x_0) - f'(x_0)h + ... \text{ considering a step in the opposite direction }
\eer
Neglecting higher order terms and simplifying equations (2) and (3) we get
\beq
0 \leq f'(x_0)h \qquad \text{and} \qquad 0 \leq -f'(x_0)h
\eeq
The above equations can be satisfied only for $f'(x_0)h=0$. Since $h\neq{0}$, we must have $f'(x_0)=0$. This is the first order optimality condition.
\subsection{Dropping the $\leq$}
If we assume $f(x_0)<f(x_0+h)$, then for this to occur we need $f'(x)<0$ on the left of $x_0$ and $f'(x)>0$ on the right of $x_0$. If we assume the $f'(x)$ is a continuous function, then $f'(x_0)=0$. As $f'(x)$ changes sign from left of $x_0$ to right of $x_0$ it needs to take all possible values between negative an positive. The only value between negative and positive is $0$. So, $f'(x_0)=0$.\
This is an application of the \textbf{intermediate value theorem}\footnote{https://en.wikipedia.org/wiki/Intermediate\_value\_theorem}. \\
\textit{In mathematical analysis, the intermediate value theorem states that if f is a continuous function whose domain contains the interval [a, b], then it takes on any given value between f(a) and f(b) at some point within the interval}

\section{Proof: second condition}
Using $f'(x_0)=0$, the Taylor expansion becomes
\beq
f(x_0+h) = f(x_0) + \frac{1}{2}f''(x_0)h^2 + ...
\eeq
Neglecting higher order terms, we note that since $f(x_0+h) \geq f(x_0)$ we must have $\frac{1}{2}f''(x_0)h^2 \geq 0$. This implies $f''(x_0) \geq 0$. If we assume $f(x_0+h) > f(x_0)$, we will get $f''(x_0) > 0$.
\end{document}
