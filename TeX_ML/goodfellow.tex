\documentclass{article}
\newcommand{\beq}{\begin{equation}}
\newcommand{\eeq}{\end{equation}}
\newcommand{\ber}{\begin{eqnarray}}
\newcommand{\eer}{\end{eqnarray}}
\newcommand{\nn}{\nonumber}
\newcommand{\dd}[2]{\frac{d}{d{#2}}{(#1)} }
\newcommand{\pdd}[2]{\frac{\partial{#1}}{\partial{#2}}}
\newcommand{\bD}{{\mathbf{D}}}
\usepackage{amsmath}
\usepackage{amsfonts}
\usepackage{bm}
\usepackage[hyphens]{url}
\usepackage{amssymb} 
\usepackage[utf8]{inputenc} 
%\usepackage[ngerman]{babel} 
\usepackage[T1]{fontenc}
\usepackage[margin=2.5cm]{geometry}
\usepackage{listings}
\usepackage{tikz}
\usepackage{cancel}
\definecolor {processblue}{cmyk}{0.96,0,0,0}
\usetikzlibrary {positioning}
\usepackage{hyperref}
\begin{document}
\title{Notes from Goodfellow}
\author{Nachiket Gokhale}
\date{\today}
\maketitle
\section{PCA Section 2.12}
Given a decoder $\pmb{D}$, what should the encoder be? The idea is that of all the points $\pmb{c}$ that one could map $\pmb{x}$ to, we should choose the point $\pmb{c}^*$ that minimizes $\|\pmb{x}-g(\pmb{c})\|_2$. Taking derivatives we find that the minimizer satisfies $\pmb{c}^* = \pmb{D}^T\pmb{x}$. So, $\pmb{D}^T$ is our encoder.  
\end{document}

