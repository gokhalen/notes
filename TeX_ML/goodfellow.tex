\documentclass{article}
\newcommand{\beq}{\begin{equation}}
\newcommand{\eeq}{\end{equation}}
\newcommand{\ber}{\begin{eqnarray}}
\newcommand{\eer}{\end{eqnarray}}
\newcommand{\nn}{\nonumber}
\newcommand{\dd}[2]{\frac{d}{d{#2}}{(#1)} }
\newcommand{\pdd}[2]{\frac{\partial{#1}}{\partial{#2}}}
\newcommand{\bD}{{\mathbf{D}}}
\usepackage{amsmath}
\usepackage{amsfonts}
\usepackage{bm}
\usepackage[hyphens]{url}
\usepackage{amssymb} 
\usepackage[utf8]{inputenc} 
%\usepackage[ngerman]{babel} 
\usepackage[T1]{fontenc}
\usepackage[margin=2.5cm]{geometry}
\usepackage{listings}
\usepackage{tikz}
\usepackage{cancel}
\definecolor {processblue}{cmyk}{0.96,0,0,0}
\usetikzlibrary {positioning}
\usepackage{hyperref}
\begin{document}
\title{Notes from Goodfellow}
\author{Nachiket Gokhale}
\date{\today}
\maketitle
\section{PCA Section 2.12}
Given a decoder $\pmb{D}$, what should the encoder be? The idea is that of all the points $\pmb{c}$ that one could map $\pmb{x}$ to, we should choose the point $\pmb{c}^*$ that minimizes $\|\pmb{x}-g(\pmb{c})\|_2$. Taking derivatives we find that the minimizer satisfies $\pmb{c}^* = \pmb{D}^T\pmb{x}$. So, $\pmb{D}^T$ is our encoder.
\section{Covariance and independence Section 3.8}
\ber
\text{ indepdent variables } & & \implies \text{ zero covariance }\\
\text{ non-zero covariance } & & \implies \text{ dependent variables }\\
\text{ zero covariance }     & & \implies \text{ no linear dependence, can be non-linear dependence }\\
\text{ dependent variables } & & \implies \text{ possible zero covariance, see below}\\
\eer
$\mathrm{x} \in [-1,1]$ be a uniform random variable. Hence, $P(x)=\frac{1}{2}$. $s$ is a discrete random variable taking values $1$ or $-1$ with probability $\frac{1}{2}$. We generate a new random variable $y=sx$. Clearly, $x$ and $y$ are not independent, because $x$ completely determines the magnitude of $y$. However, $\text{Cov}(x,y)=0$. We will prove this. \\
Clearly $x$ and $s$ are independent. Hence,
\beq
\mathbb{E}(y) = \mathbb{E}(xs) = \mathbb{E}(x)\mathbb{E}(s)
\eeq
\beq
\mathbb{E}(x) = \int_{-1}^{1}P(x)xdx = \int_{-1}^{1}\frac{1}{2}xdx = \frac{1}{2}\Big[\frac{x^2}{2}\Big]_{-1}^{1} = 0
\eeq
\beq
\mathbb{E}(s) = P(s=1)1 + P(s=-1)(-1) = 0.5*1+(0.5)*(-1) = 0
\eeq
Therefore $\mathbb{E}(y)=0$
\beq
\text{Cov}(x,y) = \mathbb{E}\Big[(x-\mathbb{E}(x))(y-\mathbb{E}(y))\Big] = \mathbb{E}\Big[xy\Big] = \mathbb{E}\Big[sx^2\Big]
\eeq
Again, since $x$ and $s$ are independent
\beq
\text{Cov}(x,y) = \mathbb{E}\Big[sx^2\Big] =  \mathbb{E}(x^2)\mathbb{E}(s)=0
\eeq
$x$ and $y$ are dependent but their covariance is zero.
\section{ Multinomial/Multinoulli distribution: 3.9.2 }
From \url{https://www.euanrussano.com/post/probability/multinoulli_multinomial/}.
\subsection{Expectation of product}
Let $x,y$ be independent random variables. Then
\ber
\mathbb{E}(xy) &=& \int \int P(x,y)xy dxy = \int\int P_1(x)P_2(y) xy dxdy \\
               &=& \int P_1(x)xdx \int P_2(y)ydy = \mathbb{E}(x)\mathbb{E}(y)
\eer
I think the critical part is that $P(x,y)=P_1(x)P_2(y)$ if $x,y$ are independent.
\subsection{Multinoulli distribution}
This distribution is also called categorial distribution, since it can be used to model events with K possible outcomes. Bernoulli distribution can be seen as a specific case of Multinoulli, where the number of possible outcomes K is 2. In machine learning, the multinoullli can used to model the expected class of one sample into a set of K classes. For instance, one may want to predict to which specie  in the set  a flower belongs based on its attribute. Then species K follow a multinoulli distribution.\\
Consider the $p(x=k)$ the probability that the sample $x$ belongs to class k. Here  $x$ could be the attributes of a flower in the example above, or one side of a die in the roll of it. If the set of classes is $K \,\in \, 1,2,3,\cdots,K$ then the probability of each outcome can be written as:
\ber
p(x=1) &=& p_1\\
p(x=2) &=& p_2\\
... & &\\
p(x=K) &=& p_K
\eer
Naturally, the probabilities sum to 1.0. ($\sum_{i}^{K}p(x=i)=1.0$). Coming back to the example of flowers classification, say that for a sample  the following probabilites where obtained for each of the 3 classes.
\ber
p(x=1) &=& 0.1\\
p(x=2) &=& 0.3\\
... & &\\
p(x=K) &=& 0.6
\eer
Clearly $\sum_{i}^{K}p(x=i)=1.0$ and one would say that the sample is most probable from class 3.
\subsection{Multinomial distribution}
The multinomial distribution describes repeated and independent Multinoulli trials. It is a generalization of he binomial distribution, where there may be K possible outcomes (instead of binary. As an example in machine learning and NLP (natural language processing), multinomial distribution models the counts of words in a document. Similar to Multinoulli, we say that a sample $x$ may take $K$ possible outcomes, each one with prabability $p_K$ after n successive trials. The probability (pmf) of a certain (particular) outcome can be modeled using the formula:
\beq
p(x=k) = \frac{n!}{x_1!x_2!\cdots x_k!}p_1^{x_1}p_2^{x_2}\cdots{p_k^{x_k}}
\eeq
Where $n$ is the number of trials, $x_i$ is the number of times event $i$ occurs and is the probability $p_i$ of event $i$ at each independent trial.\\
As an example, consider a problem which can take 3 outcomes at each trial. The probability of obtaining one specific outcomes can be written as:
\beq
p(x=k) = \frac{n!}{x_1!x_2!x_3!}p_1^{x_1}p_2^{x_2}p_3^{x_3}
\eeq
This can be used to model, for instance, the probability of one specific outcome on a chess tournment. Say that we want to determine what is the probability that, after 12 games, player 1 will have 7 wins, player 2 will have 2 wins and the remaining games will finish in draw. For that, suppose that the probability that Player 1 wins is 0.4, Player 2 is 0.35 and the tie has probability 0.25. Therefore we have,
\ber
n &=& 12\\
x_1 &=& 7\\
x_2 &=& 2\\
x_3 &=& 3 \\
p1  &=& 0.4\\
p2  &=& 0.35\\
p3  &=& 0.25
\eer
Replacing that in the formula shown above:
\ber
p(x=k) = \frac{12!}{7!2!3!}{0.4}^7{0.35}^2{0.25}^3 = 0.0248
\eer
\section{Equation 3.52}
$a$ and $c$ are independent given $b$
\beq
p(a,b,c) = p(a)p(b|a)p(c|b)
\eeq
$p(a,b,c)$ seems to be the probability of $a$ and $b$ and $c$ occurring. The rhs can be simplified to
\beq
p(a,b)p(c|b) = p(a,b)p(c|a,b)  \text{ because of independence}
\eeq
which is the same thing as $p(a,b,c)$.
\section{Equation 5.1}
\beq
p(\mathbf{x}) = p(x_1)p(x_2|x_1)p(x_3|x_1,x_2)
\eeq
RHS =
\beq
p(x_1\cap x_2)p(x_3|x_1x_2) = p(x_1\cap x_2 \cap x_3) = p(\mathbf{x})
\eeq
\section{Biased and unbiased estimators}
\subsection{Bernoulli distribution}
Random variable takes value $k=1$ with prob $\theta$ and $k=0$ with prob $1-\theta$. PMF is $\theta^k(1-\theta)^{1-k}$ assuming $0^0=1$. The true mean is
\beq
\theta = 1*P(1) + 0*P(0) = \theta
\eeq
The estimator for the mean is
\beq
\hat{\theta}_m = \frac{1}{m}\sum_{i=1}^{m}x^{(i)}
\eeq
Now,
\beq
\mathbb{E}(\hat{\theta}_m) = \frac{1}{m}\sum_{i=1}^{m}\mathbb{E}(x^{(i)})
\eeq
Now
\beq
\mathbb{E}(x^{(i)}) = 1*P(1) + 0*P(0) = \theta \text{ ... } x^{(i)} \text{ can take values 0 or 1 only. }
\eeq
So,
\beq
\mathbb{E}(\hat{\theta}_m) = \frac{1}{m}\sum_{i=1}^{m}\theta = \theta \text{ ...the true mean. Therefore this estimator is unbiased.}  
\eeq
\subsection{Biased estimator for the variance of Gaussian}
From \url{https://dawenl.github.io/files/mle_biased.pdf} Dawn Liang CMU.
The estimator
\beq
\hat{\mu}_{m} = \frac{1}{m}\sum_{i=1}^{m} x^{(i)}
\eeq
is unbiased for the mean of the Gaussian. The biased estimator for the variance of the Gaussian is
\ber
\hat{\sigma}^2_m &=& \frac{1}{m}\sum_{i=1}^{m}(x^{(i)}-\hat{\mu}_m)^2 = \frac{1}{m}\Big[\sum_{i=1}^{m}(x^{(i)})^2 - 2\hat{\mu}_{m}\sum_{i=1}^{m}x^{(i)} + \hat{\mu}_m(m)\Big]\\
&=& \frac{1}{m}\Big[\sum_{i=1}^{m}(x^{(i)})^2 - m\hat{\mu}_m^2\Big] \label{eqn:biasedgaussestim}
\eer
The expected value of this estimator is
\ber
\mathbb{E}(\hat{\sigma}^2_m) = \frac{1}{m}\Big[\sum_{i=1}^{m}\mathbb{E}((x^{(i)})^2)\Big]-\mathbb{E}(\hat{\mu}_m^2)
\eer
Using $\mathbb{E}((x^{(i)})^2)=\mathbb{E}(x^2) \,\,\forall\,\,i$
\beq
\label{eqn:expvargaussestim}
\mathbb{E}(\hat{\sigma}^2_m) = \mathbb{E}(x^2) - \mathbb{E}(\hat{\mu}^2_m)
\eeq
The true variances of $x$ and $\hat{\mu}^2$ are:
\ber
\sigma^2 = \mathbb{E}(x^2) - (\mathbb{E}(x))^2 \implies (\mathbb{E}(x^2) = \sigma^2 + \mu^2 \cdots \text{because } E(x)=\mu\\
\sigma^2_{\hat{\mu}} = \mathbb{E}(\hat{\mu}^2) - \mathbb{E}(\hat{\mu}_m)^2 \implies \mathbb{E}(\hat{\mu}^2_m) =  \sigma^2_{\hat{\mu}} + \mu^2 \cdots \text{because } E(\hat{\mu}_m)=\mu
\eer
Using the above equations in (\ref{eqn:expvargaussestim}),
\beq
\label{eqn:expvargaussestim2}
\mathbb{E}(\hat{\sigma}^2_m)  = \sigma^2 - \sigma^2_{\hat{\mu}}
\eeq
But,
\ber
\sigma^2_{\hat{\mu}_m} &=& \text{var}\Big( \frac{1}{m}\sum_{i=1}^{m} x^{(i)}\Big) \\
&=& \frac{1}{m^2}\text{var}\Big(\sum_{i=1}^{m} x^{(i)}\Big) \\
&=& \frac{1}{m^2}\sum_{i=1}^{m}\text{var}(x^{(i)}) = \frac{1}{m^2}(m\text{var}(x^{(i)}))\\
&=&\frac{1}{m}\text{var}(x^{(i)})= \frac{\sigma}{m}
\eer
Using the above equation in (\ref{eqn:expvargaussestim}),
\beq
\mathbb{E}(\hat{\sigma}^2_m) = \sigma^2 - \frac{\sigma^2}{m} = \Big(1-\frac{1}{m}\Big)\sigma^2
\eeq
Since $\mathbb{E}(\hat{\sigma}^2_m)\ne \sigma^2$, equation (\ref{eqn:biasedgaussestim}) is not an unbiased estimator for the variance. 
\end{document}

