\documentclass{article}
\newcommand{\beq}{\begin{equation}}
\newcommand{\eeq}{\end{equation}}
\newcommand{\ber}{\begin{eqnarray}}
\newcommand{\eer}{\end{eqnarray}}
\newcommand{\nn}{\nonumber}
\newcommand{\dd}[2]{\frac{d}{d{#2}}{(#1)} }
\usepackage{graphicx}
\usepackage{amsmath}
\usepackage{amsfonts}
\usepackage{url}
\begin{document}
\title{Precision Recall}
\section{Precision recall}
From Geron pg 87. Increasing threshold decreases recall and increases precision.Lowering threshold increases recall and reduces precision. (from Notes this is not strictly necessary - recall may stay constant - \url{https://scikit-learn.org/stable/auto_examples/model_selection/plot_precision_recall.html})\\

increasing precision reduces recall, and vice versa. This is called the precision/recall tradeoff.

Best statement is from professor: ``In slides, the keyword is "must". It is not a must for P to increase and R to decrease.''

\section{AUC Curve for a random classifier}
See: ``auc\_curve\_for\_random\_classifier.py''

\end{document}
