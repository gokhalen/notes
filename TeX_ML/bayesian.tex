\documentclass{article}
\newcommand{\beq}{\begin{equation}}
\newcommand{\eeq}{\end{equation}}
\newcommand{\ber}{\begin{eqnarray}}
\newcommand{\eer}{\end{eqnarray}}
\newcommand{\nn}{\nonumber}
\newcommand{\dd}[2]{\frac{d}{d{#2}}{(#1)} }
\newcommand{\pdd}[2]{\frac{\partial{#1}}{\partial{#2}}}
\newcommand{\bw}{{\mathbf{w}}}
\newcommand{\bff}{\mathbf{f}}
\newcommand{\bPi}{\mathbf{\Pi}}
\newcommand{\bQ}{{\mathbf{Q}}}
\newcommand{\bq}{{\mathbf{q}}}
\newcommand{\bK}{{\mathbf{K}}}
\newcommand{\bU}{{\mathbf{U}}}
\newcommand{\bS}{{\mathbf{S}}}
\newcommand{\bV}{{\mathbf{V}}}

\usepackage{amsmath}
\usepackage{amsfonts}
\usepackage[hyphens]{url}
\usepackage{amssymb} 
\usepackage[utf8]{inputenc} 
%\usepackage[ngerman]{babel} 
\usepackage[T1]{fontenc}
\usepackage[margin=2.5cm]{geometry}
\usepackage{listings}
\usepackage{tikz}
\usepackage{cancel}
\definecolor {processblue}{cmyk}{0.96,0,0,0}
\usetikzlibrary {positioning}
\usepackage{hyperref}
\begin{document}
\title{Bayesian statistics in Python}
\author{Nachiket Gokhale}
\date{\today}
\maketitle
\section{Bayes theorem}
Source: \url{https://analyticsindiamag.com/a-guide-to-bayesian-statistics-in-python-for-beginners/}
Bayes theorem:
\beq
P(A|B)P(B) = P(B|A)P(A) 
\eeq
Proof: Draw Venn Diagrams
\ber
P(A|B) = \frac{n_{A\cap{B}}}{n_B} \implies P(A|B)n_B = n_{A\cap{B}}\\
P(B|A) = \frac{n_{A\cap{B}}}{n_A} \implies P(B|A)n_A = n_{A\cap{B}}
\eer
Above equations imply
\ber
P(A|B)n_B = P(B|A)n_B
\eer 
Divide both sides by n and use $P(A)=n_A/n$ and $P(B)=n_B/n$
\beq
\label{eqn:bayesidentity}
P(A|B)\frac{n_B}{n} = P(B|A)\frac{n_B}{n} \implies P(A|B)P(B) = P(B|A)P(B)
\eeq
\section{Prior probability}
$P(A)$ and $P(B)$ are prior probabilities of $A$ and $B$. These are global, universal probabilities, without taking into account any additional evidence.
\section{Likelihood function and posterior probability}
\ber
P(A|B) = \frac{P(B|A)P(A)}{P(B)}, \qquad P(B|A) = \frac{P(A|B)P(B)}{P(A)}
\eer
In the formula on the left $P(A|B)$ is the posterior, $P(B|A)$ is the likelihood.\\
In the formula on the right $P(B|A)$ is the posterior, $P(A|B)$ is the likelihood.
\section{An identity}
The problem solved in the above mentioned link is: For example, suppose we have 2 buckets A and B. 
In bucket A we have 30 blue balls and 10 yellow balls, while in bucket B we have 20 blue and 20 yellow balls.
We are required to choose one ball. We have picked a blue ball. What is the chance that we have picked 
it from bucket A? A simple solution is There are 50 blue balls. 30 in bucket A and 20 in bucket B
So, given that we have picked a blue ball the probability we have picked it from A is 30/50 = 0.6\\

The link above solves this using the formula below, which is not obvious. Therefore we explain it.

\ber
\frac{P(Blue|A)P(A)}{P(Blue|A)P(A) + P(Blue|B)P(B)} = P(A/Blue)
\eer

Using equation (\ref{eqn:bayesidentity}) the denominator of the LHS becomes

\beq
P(A|Blue)P(Blue) + P(B|Blue)P(Blue) = P(Blue)*\cancelto{1}{(P(A|Blue) + P(B|Blue))}
\eeq
The second term is $1$ because once a blue ball has been selected, it has to come from $A$ or $B$. So, the denominator becomes $P(Blue)$. The LHS simplifies to
\beq
\frac{P(Blue|A)P(A)}{P(Blue)} = \frac{P(A|Blue)P(Blue)}{P(Blue)} = P(A|Blue)
\eeq
\section{Analytics India Mag Problem: SE post}
Related SO post: \url{https://math.stackexchange.com/questions/4288910/bayes-theorem-selecting-balls}
I'm trying to solve the problem defined here: \url{https://analyticsindiamag.com/a-guide-to-bayesian-statistics-in-python-for-beginners/} but I cannot understand the explanation of the second part. The first part of the problem, which is:

\textbf{Part 1}
For example, suppose we have 2 buckets A and B. In bucket A we have 30 blue balls and 10 yellow balls, while in bucket B we have 20 blue and 20 yellow balls. We are required to choose one ball. What is the chance that we choose bucket A?


{\textbf{Solution}} Let event $A$ be selecting bucket A. Let event $B$ be selecting bucket B. Let $Blue$ be the event of selcting blue ball and $Yellow$ be the event of selecting the yellow ball.  We are asked to find $P(A|Blue)$. By Bayes' theorem

\beq
P(A|Blue) = \frac{P(Blue|A)P(A)}{P(Blue)}
\eeq

Since a blue ball has to be picked from bucket A or B we can write

\beq
P(Blue) = P(Blue|A)P(A) + P(Blue|B)P(B)
\eeq

So our equation becomes

\ber
P(A|Blue) &=& \frac{P(Blue|A)P(A)}{P(Blue|A)P(A) + P(Blue|B)P(B)}\\
          &=& \frac{(3/4)*(1/2)}{(3/4)*(1/2)+(1/2)*(1/2)} \\
          &=& 0.6
\eer

Similarly,

\ber
P(B|Blue)   &=& 0.4 \\
P(A|Yellow) &=& \frac{1}{3}\\
P(B|Yellow) &=& \frac{2}{3}
\eer



{\textbf{Part 2}} We put back the ball chosen in Part 1. We choose another ball again, this time it turns out to be yellow. What is the probability that both times we choose the ball from bucket A?.

{\textbf{This, I do not understand}}.

It seems to me that we have two events:\\
Event $C$: A | Yellow  (Bucket A given ball is Yellow) \\
Event $D$: A | Blue    (Bucket B given ball is Blue)\\
And so we have\\
\ber
P(C) = P(A|Yellow) = \frac{1}{3} \\
P(D) = P(A|Blue)   = 0.6
\eer

And it seems that we are interested in $P(C|D)$ which is, by Bayes' theorem
\beq
P(C|D) = \frac{P(D/C)P(C)}{P(D)}
\eeq
And I have no idea what to do next.\\

It seems that they are doing 
\ber
P(C|D)&=&P(A|Yellow\Big{|}A|Blue)\\
&=& \frac{P(Yellow|A)P(A|Blue)}{P(Yellow)} \\
&=& \frac{0.25*0.6}{P(Yellow|A)P(A|Blue)+P(Yellow|B)P(B|Blue)} \\
      &=& \frac{0.15}{0.25*0.6+0.5*0.4} \\
      &=& \frac{0.15}{0.35} \\
      &=& 0.4285
\eer
But I cannot understand why.
\section{ Graham's Answer}
Do not define events as below:\\
Event $C$: A | Yellow  (Bucket A given ball is Yellow) \\
Event $D$: A | Blue    (Bucket B given ball is Blue)\\

\noindent{Define them like this}:\\
$A$: Selecting bucket A \\
$B$: Selecting blue ball \\
$Y$: Selecting yellow ball \\

\noindent Events are actions. Do not define an event as selecting bucket A given Yellow.\\

For Part 1:

\ber
P(A|B) &=& \frac{P(B|A)P(A)}{P(B)} \\
&=& \frac{P(B|A)P(A)}{P(B|A)P(A) + P(B|A^\complement)P(A^\complement)} \qquad \text{ where  } A^\complement \text{ is the complement of } A \\
&=& \frac{\frac{3}{4}\frac{1}{2}}{\frac{3}{4}\frac{1}{2}+\frac{2}{4}\frac{1}{2}}\\
&=& \frac{3}{5}
\eer

For Part 2: $P(A|Y,B)$ means the probability of $A$ given $Y$, given that we are in the space in which $B$ has occurred.

\ber
P(A|Y,B) &=& \frac{P(Y|A,B)P(A|B)}{P(Y|B)} \\
&=& \frac{P(Y|A,B)P(A|B)}{P(Y|A,B)P(A|B) + P(Y|A^\complement,B)P(A^\complement|B)}
\eer
We know $P(Y|A,B)=P(Y|A)=1/4$
\ber
P(A|Y,B) &=& \frac{\frac{1}{4}\frac{3}{5}}{\frac{1}{4}\frac{3}{5} + \frac{2}{4}\frac{2}{5}} = \frac{3}{7}
\eer
\end{document}

