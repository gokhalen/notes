\documentclass{article}
\newcommand{\beq}{\begin{equation}}
\newcommand{\eeq}{\end{equation}}
\newcommand{\ber}{\begin{eqnarray}}
\newcommand{\eer}{\end{eqnarray}}
\newcommand{\nn}{\nonumber}
\newcommand{\dd}[2]{\frac{d}{d{#2}}{(#1)} }
\usepackage{amsmath}
\usepackage{amsfonts}
\usepackage[hyphens]{url}
\usepackage{amssymb} 
\usepackage[utf8]{inputenc} 
%\usepackage[ngerman]{babel} 
\usepackage[T1]{fontenc}
\usepackage[margin=2.5cm]{geometry}
\usepackage{listings}
\usepackage{hyperref}
\begin{document}
\title{Filter maximization}
\author{Nachiket Gokhale}
\date{\today}
\maketitle
\section{Introduction}
In Chollet's Deeplearning book in Section 5.4.2, 'Visualizing Filters', we define the loss as \\

loss = K.mean(layer\_output[:, :, :, filter\_index]) \\

This loss represents the strength of the layer output \\

After, computing the gradient of this loss, we \textit{maximize} it. The trick lies in this line:\\

input\_img\_data += grads\_value * step\\

To minimize, the update has a -ve sign, for maximization the sign is positive.

Since the convolutional layer weights can be negative, leading to negative values in layer\_output, we might be perhaps better off maximizing:\\

loss = K.mean(layer\_output[:, :, :, filter\_index]*layer\_output[:,:,:,filter\_index]) \\

or if we are not worried about differentiability, \\

loss = K.abs(K.mean(layer\_output[:, :, :, filter\_index]))






\end{document}
