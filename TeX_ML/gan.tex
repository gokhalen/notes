\documentclass{article}
\newcommand{\beq}{\begin{equation}}
\newcommand{\eeq}{\end{equation}}
\newcommand{\ber}{\begin{eqnarray}}
\newcommand{\eer}{\end{eqnarray}}
\newcommand{\nn}{\nonumber}
\newcommand{\dd}[2]{\frac{d}{d{#2}}{(#1)} }
\usepackage{amsmath}
\usepackage{amsfonts}
\usepackage{multirow}
\usepackage{url}
\begin{document}
\title{GANs}
\author{Nachiket Gokhale}
\date{\today}
\maketitle
\section{Objective Function}
The IISc notes give the objective function for a GAN as
%
\beq
\label{eqn:objfunc}
\pi = E_x[\log(D(x))] + E_z[\log(1-D(G(z)))]
\eeq
Drop the expectations (to make things simpler) to get,
\beq
\label{eqn:objfunc2}
\pi =\log(D(x)) + \log(1-D(G(z)))
\eeq
%
$D(x)$ is the discriminators estimate of the probabuility that the real data instance x is real. $E_x$ is expected value over all real instances. $E_z$ is the expeced value over all random inputs to the generator. D(G(z)) is the discriminator's estimate of the probability that a fake instance is real.\\

The discriminator wants to maximize \ref{eqn:objfunc2}. To see this, if $D(x)$ is always zero, then $\pi\,\rightarrow\,-\infty$. To see this, if $D(x)$ is always 1, then $\pi\,\rightarrow\,-\infty$. Since $D(x)$ is a probability it lies between zero and (1). So the intuition is that if we try to maximize  \ref{eqn:objfunc2} then $D$ will learn what is true and what is fake. \\

Now for the generator $G$. What the generator wants to do is to fool the discriminator. It wants make $D = 1$ all the time. So it wants to make $\pi\rightarrow{-\infty}$.\\

So to train $D$ we need to maximize \ref{eqn:objfunc2} and to train $G$ we need to minimize \ref{eqn:objfunc2}.\

\end{document}


