\documentclass{article}
\newcommand{\beq}{\begin{equation}}
\newcommand{\eeq}{\end{equation}}
\newcommand{\ber}{\begin{eqnarray}}
\newcommand{\eer}{\end{eqnarray}}
\newcommand{\nn}{\nonumber}
\newcommand{\dd}[2]{\frac{d}{d{#2}}{(#1)} }
\newcommand{\pdd}[2]{\frac{\partial}{\partial{#2}}{(#1)} }
\usepackage{amsmath}
\usepackage{amsfonts}
\usepackage{url}
\begin{document}
\title{Gradient normal to level surface}
\author{Nachiket Gokhale}
\date{\today}
\maketitle
\section{Gradient normal to level surface}
See: \url{https://ocw.mit.edu/courses/mathematics/18-02sc-multivariable-calculus-fall-2010/2.-partial-derivatives/part-b-chain-rule-gradient-and-directional-derivatives/session-36-proof/MIT18_02SC_notes_19.pdf} Basically, let 
\beq
f(x(t),y(t),z(t)) = c
\eeq
be a level surface of $f$. Differentiate
\beq
\pdd{f}{x}\dd{x(t)}{t} + \pdd{f}{y}\dd{y(t)}{t} + \pdd{f}{z}\dd{z(t)}{t} = 0
\eeq
Write it as a dot product of two vectors.
\ber
\Big{(}\pdd{f}{x},\pdd{f}{y},\pdd{f}{z}\Big{)}{\bullet}\Big{(}\dd{x(t)}{t},\dd{y(t)}{t},\dd{z(t)}{t}\Big{)} = 0\\
\nabla{f}\bullet{\dd{\mathbf{r}(t)}{t}} = 0
\eer
So, the gradient of $f$ is perpendicular to the tangent to the curve $\dd{\mathbf{r}(t)}{t}$. This is exactly what we wanted to prove. In other words, gradient is tangent to the level surface.
\end{document}
