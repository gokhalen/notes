\documentclass{article}
\newcommand{\beq}{\begin{equation}}
\newcommand{\eeq}{\end{equation}}
\newcommand{\ber}{\begin{eqnarray}}
\newcommand{\eer}{\end{eqnarray}}
\newcommand{\nn}{\nonumber}
\newcommand{\dd}[2]{\frac{d}{d{#2}}{(#1)} }
\newcommand{\pdd}[2]{\frac{\partial{#1}}{\partial{#2}}}
\usepackage{amsmath}
\usepackage{amsfonts}
\usepackage[hyphens]{url}
\usepackage{amssymb} 
\usepackage[utf8]{inputenc} 
%\usepackage[ngerman]{babel} 
\usepackage[T1]{fontenc}
\usepackage[margin=2.5cm]{geometry}
\usepackage{listings}
\usepackage{tikz}
\definecolor {processblue}{cmyk}{0.96,0,0,0}
\usetikzlibrary {positioning}
\usepackage{hyperref}
\begin{document}
\title{Backpropagation}
\author{Nachiket Gokhale}
\date{\today}
\maketitle
\section{Principal Component Analysis}
As per Andrew Ng's notes, PCA goes like the following. First take the features $x^{(i)} \in \mathbb{R}^n$ which are assumed to have zero mean and and also perhaps scaled by the variance/standard deviation and construct the matrix

\ber
\Sigma = \frac{1}{m}\sum_{i=0}^{m}x^{(i)}{x^{(i)}}^{T}
\eer
%
SVD of $\Sigma$ gives
\beq
\Sigma = USV^{T} \qquad U,S,V \in \mathbb{R}^{n\times{n}}
\eeq
Pick the first $k$ columns of $U$:
\beq
\hat{U} = U[:,0:k]  \qquad \hat{U} \in \mathbb{R}^{n\times{k}}
\eeq
Define new variables:
\beq
z^{(i)}   = {\hat{U}}^{T}x^{(i)} \qquad z^{(i)} \in \mathbb{R}^{k},\,\, {\hat{U}}^{T} \in \mathbb{R}^{k\times{n}}
\eeq
Now training data becomes $(z^{(0)},y^{(0)}), (z^{(1)},y^{(1)}), \cdots, (z^{(m)},y^{(m)})$. If needed, the original variables can be recovered using
\beq
x^{(i)} = \hat{U}z^{(i)}
\eeq

%
%
\subsection{Aside} 
The matrix $\Sigma$ is symmetric-positive-semi-definite. To see this define:
\beq
a_{pq}^{(i)} = x^{(i)}_{p}{x^{(i)}_{q}}
\eeq
We have:
\ber
y_{p}a_{pq}^{(i)}y_{q} &=& y_{p}x^{(i)}_{p}{x^{(i)}_{q}}y_{q}  \\
&=& (y_{c}x^{(i)}_{c})({x^{(i)}_{c}}y_{c}) \text{ ... switching dummy inidces to }c\\
&=& (y_{c}x^{(i)}_{c}) \ge 0
\eer
Since, $\Sigma$ is a sum of all $a^{i}$, it is also symmetric-positive-semi-definite.
\section{Model reduction in mechanics}
We are typically interested in solving the system
\beq
\label{eqn:kxeqf}
Kx = f \qquad \text{where } K\in\mathbb{R}^{n\times{n}} \text{ and } x,f \in \mathbb{R}^n
\eeq
$K$ is symmetric positive definite with distinct eigenvalues. We typically solve for $k$ eigenvectors of $K$ with $k \ll n$. We construct a matrix $\phi$ whose columns are the $k$ eigenvectors of $K$. So $\phi \in \mathbb{R}^{n\times{k}}$. We define new variables $y$
\beq
x = \phi{y} \qquad \text{ or equivalently } y = \phi^Tx
\eeq
This is possible because the eigenvectors of symmetric positive definite matrices with distinct eigenvalues are orthogoal. This gives
\beq
\phi^Tx = \phi^T\phi y = y \qquad \text{ since } \phi^T\phi = \mathbb{I}^{k\times{k}}, \qquad \mathbb{I} \text{ is the identity matrix.}
\eeq
Using the above equation in equation (\ref{eqn:kxeqf}) we get
\beq
K\phi{y} = f
\eeq
Multiplying by $\phi^T$ gives a square $k\times{k}$ system
\beq
\phi^TK\phi{y} = \phi^Tf
\eeq
\subsection{Aside 1}
Saying $x=\phi{y}$ is equivalent to saying $x=y_{1}\phi_{1}+\cdots+y_{k}\phi_{k}$ where $\phi_m$ is the $m^{\text{th}}$ column of $\phi$, i.e. the $m^{\text{th}}$ eigenvector and $y_m$ is the $m^{\text{th}}$ component of $y$. This is the conventional expansion of $x$ using the significant eigenfunctions of $K$. To see this, consider the following small example
\ber
\begin{pmatrix}
  \phi_{00} & \phi_{01} \\
  \phi_{10} & \phi_{11} \\
  \phi_{20} & \phi_{21} \\
  \phi_{30} & \phi_{31} \\
\end{pmatrix}
\begin{pmatrix}
  y_0 \\
  y_1
\end{pmatrix}
=\begin{pmatrix}
  \phi_{00}y_0 + \phi_{01}y_1 \\
  \phi_{10}y_0 + \phi_{11}y_1 \\
  \phi_{20}y_0 + \phi_{21}y_1 \\
  \phi_{30}y_0 + \phi_{31}y_1\\
\end{pmatrix}
=y_0\begin{pmatrix}
  \phi_{00}  \\
  \phi_{10}  \\
  \phi_{20}  \\
  \phi_{30}  \\
\end{pmatrix}
+
y_1\begin{pmatrix}
  \phi_{01}  \\
  \phi_{11}  \\
  \phi_{21}  \\
  \phi_{31}  \\
\end{pmatrix}
\eer
\subsection{Aside 2}
Let $K\in\mathbb{R}^{n\times{n}} $ be a symmetric positive definite matrix with $n$ distinct eigenvalues. Let $x$ and $y$ be eigenvectors associated with $\lambda_x$ and $\lambda_y$. We have
\beq
Kx=\lambda_xx \implies y^TKx = \lambda_xy^Tx
\eeq
But,
\beq
y^TKx = (Ky)^Tx = (\lambda_yy)^Tx = \lambda_yy^Tx
\eeq
We must have
\beq
\lambda_xy^Tx = \lambda_yy^Tx
\eeq
Since $\lambda_x\ne\lambda_y$, we must have $y^Tx=0$.
This gives rise to the identity we have used before:
\beq
\phi^T\phi = \mathbb{I}^{k\times{k}}
\eeq
\end{document}

