\documentclass{article}
\newcommand{\beq}{\begin{equation}}
\newcommand{\eeq}{\end{equation}}
\newcommand{\ber}{\begin{eqnarray}}
\newcommand{\eer}{\end{eqnarray}}
\newcommand{\nn}{\nonumber}
\usepackage{amsmath}
\usepackage{amsfonts}
\usepackage{url}
\begin{document}
\title{ML Notes}
\author{Nachiket Gokhale}
\date{\today}
\maketitle
\section{Points}
\begin{enumerate}
  \item{Apply feature scaling  after spliting data set into training and test set
    \url{https://www.udemy.com/course/machinelearning/learn/lecture/19048226#notes}. This is because we do not want to learn the normalization parameters  (mean of data set or standard deviation of data set) from the test set or cross validation set. The normalization parameters need to be learned only from the test set}
  \item{Estimating the p-value for linear regression: Elements of statistical learning Hastie, Tibshirani, Friedman: Pg 47,48}
  \item{p-value: Probability of observing the given results assuming that the null-hypothesis is correct}
  \item{null-hypothesis:In inferential statistics, the null hypothesis (often denoted H0,[1]) is a general statement or default position that there is no relationship between two measured phenomena or no association among groups}
\end{enumerate}
\subsection{SVM: Optimization}
The SVM optimization problem on Wikipedia (\url{https://en.wikipedia.org/wiki/Support_vector_machine#Regression}) is\\\\
Find: $w_i, b$ that minimize
\beq
\pi = \frac{1}{2}\|\bf{w}\|^2
\eeq
Subject to:
\beq
|y_i -  \langle {\bf{w}},{\bf{x}_i} \rangle - b | \le \epsilon
\eeq
Here, $b$ does not enter into $\pi$. So, it is reasonable to ask: How can one find a variable that does not appear in the objective function?. I think the answer is: $b$ appears in the constraints. And the constraints get included in the Lagrangian, along with slack variables etc.
\section{Loss functions}
In TensorFlow/Keras the loss function `binary\_crossentropy` and activation `sigmoid` should be used for binary classification and `categorical\_crossentropy` and `softmax` for multiclass classification.
\section{Repeatable results}
Network weights are randomly initialized to break symmetries. This random initialization can result in non-reproducible results. To get repeatable weights, we need to set the seed. This can be done using `tf.random.set\_seed(some\_constant\_value)`
\section{Links}
\begin{enumerate}
\item{\url{http://neuralnetworksanddeeplearning.com/}}
\item{\url{https://stats.stackexchange.com/questions/154879/a-list-of-cost-functions-used-in-neural-networks-alongside-applications}}
\item{\url{https://www.udemy.com/course/linear-regression-with-artificial-neural-network/}}
\item{\url{https://adeshpande3.github.io/A-Beginner%27s-Guide-To-Understanding-Convolutional-Neural-Networks/} 9 Deeplearning  papers you need to know about}
\item{\url{http://cs231n.stanford.edu/} }
\item{\url{https://rdipietro.github.io/friendly-intro-to-cross-entropy-loss/}}
\end{enumerate}
\section{ML Strategy}
ML Strategy
0) ML to classify stiffness fields. Given a stiffness field, classify it into predefined classes
1) Binary classification: Is there an inclusion or not? Multiple inclusions, multiple values of stiffness. 
2) Multi-class classification: Divide training examples into 10 classes (1 homogeneous + 9 examples corresponding to grid of stiffnesses)
2.5) Input multiple displacement fields
3) Get ML to compute location, radius, and stiffness of tumor. i.e. make real vaued predictions instead of only classifying
4) Get ML to predict the complete distribution of stiffness

6) ML breast model 2D (realistic material parameters and geometry). How do we feed this realistic geometry into the NN? One way is to make a rectangular image using PyVista and then run CNN on the image. I don't like this because we're losing control on information.  
7) Consider the case of tumors spanning more than 1 training cell
\section{Other ideas}
0) See if one can output a bounding box for the stiffness. Maybe a boundary/edge detection algorithm will be able to create a non linear boundary as well.
\section{Read Papers}
\item{Oberai,Gu,Insana}
\end{document}
