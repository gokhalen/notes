\documentclass{article}
\newcommand{\beq}{\begin{equation}}
\newcommand{\eeq}{\end{equation}}
\newcommand{\ber}{\begin{eqnarray}}
\newcommand{\eer}{\end{eqnarray}}
\newcommand{\nn}{\nonumber}
\newcommand{\dd}[2]{\frac{d}{d{#2}}{(#1)} }
\usepackage{amsmath}
\usepackage{amsfonts}
\usepackage{multirow}
\usepackage{url}
\begin{document}
\title{Classifier notes}
\author{Nachiket Gokhale}
\date{\today}
\maketitle
\section{Random binary classifier}
Chollet in his book Deep Learning with Python, on page 82 says: 'With a balanced binary classification
problem, the accuracy reached by a purely random classifier would be 50\%.'. This was not obvious to me.\\

So basically we have a vector ${\bf{x}}$ of large length $n$, which contains either $0$ or $1$ with 50\% probability. This is our label vector, which is given. We now guess a vector ${\bf{y}}$, which contains either $0$ or $1$ with 50\% probability. We want to know how many entries in ${\bf{x}}$ and ${\bf{y}}$ match. That is we are interested in
\beq
\sum_{i=1}^{i=n} (x_{i} == y_{i}), 
\eeq
where $x_{i} == y_{i}$ yields $1$ if true, and $0$ otherwise.\\

Without loss of generality, rearrange ${\bf{x}}$ so that the first half contains $0$ and the last half contains $1$.  Now, when we guess $\mathbf{y}$ for the first half of ${\mathbf{x}}$, we are going to get 50\% entries 0 and the other 50\% 1. So we get 50\% entries correct for the first half of ${\bf{x}}$. Similarly for the second half of ${\mathbf{x}}$ we are going to get 50\% entries correct and 50\% entries wrong. Combining the two halves, we are going to get 50\% entries correct.

\section{Confusion matrix for linear classifier}
Suppose we are classifying examples based on some parameter $p$ which lies in $[0,1]$. If $p>c$ then the example is labelled $+ve$ and if $p\le{c}$ the example is labelled $-ve$. Let there be a very large number of examples $N$ and let them be uniformly distributed in $p$. So there are $N(1-c)$ $+ve$ examples and $Nc$ $-ve$ examples.\\
\textbf{Case 1}: Now, if our threshold is some $t > c$. i.e. if $p>t$ the examples is classified as $+ve$ and if $p\le{t}$ the example is classified as negative. So the number of examples classified $+ve$ by our classifier are $N(1-t)$ and the number of examples classified $-ve$ are $Nt$. Since $t>c$ all of our $+ve$ examples are \textit{true positives} and there are no \textit{false positives}. The number of \textit{true negatives} is $Nc$ and the number of \textit{false negatives} is $t-c$. This leads to the confusion matrix in table $(\ref{tab:confcase1})$.
\begin{table}
\begin{center}
  \begin{tabular}{cccc}
    \cline{3-4}
    & & \multicolumn{2}{|c|}{Actual class}  \\
    \cline{3-4}
    & & \multicolumn{1}{|c|}{$+ve$, $N(1-c)$} &  \multicolumn{1}{|c|}{$-ve$, $Nc$}\\
    \hline
    \multicolumn{1}{|c}{Predicted} & \multicolumn{1}{|c|}{$+ve$,$N(1-t)$} & \multicolumn{1}{|c|}{$N(1-t)$} & \multicolumn{1}{|c|}{$0$} \\
    \cline{2-4}
    \multicolumn{1}{|c}{class} & \multicolumn{1}{|c|}{$-ve$,$Nt$} & \multicolumn{1}{|c|}{$N(t-c)$} & \multicolumn{1}{|c|}{$Nc$}\\
    \hline
  \end{tabular}
\end{center}
\caption{\label{tab:confcase1} Confusion matrix for Case 1.}
\end{table}\\
%
\textbf{Case 2}:Now, if our threshold is some $t < c$. i.e. if $p>t$ the examples is classified as $+ve$ and if $p\le{t}$ the example is classified as negative. So the number of examples classified $+ve$ by our classifier are $N(1-t)$ and the number of examples classified $-ve$ are $Nt$. Since $t<c$ all of our $-ve$ examples are \textit{true negatives} and there are no \textit{false negatives}. The number of \textit{true positives} is $Nc$ and the number of \textit{false positives} is $N(c-t)$. This leads to the confusion matrix in table $(\ref{tab:confcase2})$.
\begin{table}
\begin{center}
  \begin{tabular}{cccc}
    \cline{3-4}
    & & \multicolumn{2}{|c|}{Actual class}  \\
    \cline{3-4}
    & & \multicolumn{1}{|c|}{$+ve$, $N(1-c)$} &  \multicolumn{1}{|c|}{$-ve$, $Nc$}\\
    \hline
    \multicolumn{1}{|c}{Predicted} & \multicolumn{1}{|c|}{$+ve$,$N(1-t)$} & \multicolumn{1}{|c|}{$N(1-c)$} & \multicolumn{1}{|c|}{$N(c-t)$} \\
    \cline{2-4}
    \multicolumn{1}{|c}{class} & \multicolumn{1}{|c|}{$-ve$,$Nt$} & \multicolumn{1}{|c|}{$0$} & \multicolumn{1}{|c|}{$Nt$}\\
    \hline
  \end{tabular}
\end{center}
\caption{\label{tab:confcase1} Confusion matrix for Case 2.}
\end{table}\\
\end{document}
