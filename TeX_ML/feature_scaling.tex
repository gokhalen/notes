\documentclass{article}
\newcommand{\beq}{\begin{equation}}
\newcommand{\eeq}{\end{equation}}
\newcommand{\ber}{\begin{eqnarray}}
\newcommand{\eer}{\end{eqnarray}}
\newcommand{\nn}{\nonumber}
\usepackage{amsmath}
\usepackage{amsfonts}
\usepackage{url}
\begin{document}
\title{Need for feature scaling}
\author{Nachiket Gokhale}
\date{\today}
\maketitle
\section{Least squares problem}
\ber
\text{ minimize }\nn\\
\pi(\theta_0,\theta_1,\theta_2) &=& \frac{1}{2}(\theta_0 + \theta_1x_1 + \theta_2x_2 - y)^2\\
\frac{\partial\pi}{\partial\theta_0} &=& (\theta_0 + \theta_1x_1 + \theta_2x_2 - y) \nn \\
\frac{\partial\pi}{\partial\theta_1} &=& (\theta_0 + \theta_1x_1 + \theta_2x_2 - y)x_1 \nn \\
\frac{\partial\pi}{\partial\theta_2} &=& (\theta_0 + \theta_1x_1 + \theta_2x_2 - y)x_2 \nn \\
\eer
Clearly, if $x_1$ and $x_2$ are on vastly different scales, the gradients will be on vastly different scales, and optimization will prefer one direction over the other. Therefore we need $x_1\approx x_2$
%
\section{Feature scaling and regularization}
Feature scaling is important for regularization because, feature scaling ensures that the parameters are affected about the same. (intuitive explanation, not rigorous). See \url{https://stats.stackexchange.com/questions/189176/why-do-we-need-to-normalize-data-before-applying-penalizing-methods-in-the-frame}. ``The reason to normalise your variables beforehand is to ensure that the regularisation term $\lambda$ regularises/affects the variable involved in a (somewhat) similar manner.''
%
The regularized least squares solution is 
\beq
\pmb{\theta}=(\pmb{X}^T\pmb{X} + \alpha\pmb{1})^{-1}\pmb{X}^T\pmb{y}
\eeq
From this we can see that if the features have vastly different scales, then $\alpha$ would affect them very differently. Small features would be heavily regularized, strong features very weakly so.

\end{document}
