\documentclass{article}
\newcommand{\beq}{\begin{equation}}
\newcommand{\eeq}{\end{equation}}
\newcommand{\ber}{\begin{eqnarray}}
\newcommand{\eer}{\end{eqnarray}}
\newcommand{\nn}{\nonumber}
\newcommand{\dd}[2]{\frac{d}{d{#2}}{(#1)} }
\usepackage{amsmath}
\usepackage{amsfonts}
\usepackage{url}
\usepackage{amssymb} 
\usepackage[utf8]{inputenc} 
%\usepackage[ngerman]{babel} 
\usepackage[T1]{fontenc}
\usepackage[margin=2.5cm]{geometry}
\usepackage{listings}
\usepackage{hyperref}
\begin{document}
\title{Recurrent neural networks}
\author{Nachiket Gokhale}
\date{\today}
\maketitle
\section{Introduction}
This is an exposition of Recurrent Neural Networks in Chollet pg 203. The input $I$ is of shape $(n_{batchsize},n_{lookback},n_{variables})$. This means that we have $n_{batchsize}$ time series, each with data at $n_{lookback}$ time-points for each of the $n_{variables}$. For each time series, the recurrent layer marches through $n_{lookback}$ time steps. The recurrent layer marches the output through time using the following equations:
\beq
y_{t+1} = \text{activation}(U_0\cdot{s_t} + W_0\cdot{x_t} + C_t\cdot{V_0} + b_{0}).
\eeq
Here $U_0,W_0,C_t \in \mathbb{R}^{n_{output}\times{n_{variables}}}$, $b_0 \in \mathbb{R}^{n_{output}}$, are internal weights which will be learned. $s_{t},x_{t} \in \mathbb{R}^{n_{variables}}$ are internal state and input data respectively and '$\cdot$' is a matrix-vector product. Other quantities will be marched through time using appropriate update equations. The above equation generates data which can be put into the shape $(n_{batchsize},n_{lookback},n_{output})$. If recurrent layers are stacked, then the full array is passed on to the next layer. Otherwise only the output at the final time is extracted resulting in an array of shape $(n_{batchsize},1,n_{output})$ which can be reshaped to $(n_{batchsize},n_{output})$.

\section{Input Shape}
See the source below for an example of a RNN
\begin{center}
  \lstinputlisting[language=Python]{recurrent.python}
\end{center}

\section{Same input for all units}
If the same input is applied to all units of a recurrent layer, then, in general, the output of the $n^{th}$ unit is not the same as the output of the $(n-1)^th$ unit. This is because, even though the input is the same, the latent state input for the unit $n$ is not the same as the latent state input for the unit $n-1$
 
\end{document}
