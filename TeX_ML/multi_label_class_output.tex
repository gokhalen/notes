\documentclass{article}
\newcommand{\beq}{\begin{equation}}
\newcommand{\eeq}{\end{equation}}
\newcommand{\ber}{\begin{eqnarray}}
\newcommand{\eer}{\end{eqnarray}}
\newcommand{\nn}{\nonumber}
\newcommand{\dd}[2]{\frac{d}{d{#2}}{(#1)} }
\usepackage{amsmath}
\usepackage{amsfonts}
\usepackage{multirow}
\usepackage{url}
\begin{document}
\title{Multilabel, Multiclass, Multioutput}
\author{Nachiket Gokhale}
\date{\today}
\maketitle
%
\section{Multilabel}
From Geron's chapter on Classification.\\
Facial recognition. From a photograph identify Alice,Bob,Charlie. The output vector would be [0,1,1].
%
\section{Multiclass}
%
Typically the classes are mutually exclusive. e.g. classifying digits in the MNIST dataset. We will produce an output vector containing probabilities using softmax.
%
\section{Multioutput}
Predict multiple labels where each label can be multiclass. e.g. in a photgraph predict whether bicycles, motorbikes and cars are present. And then identify the make of each bicycle,motorbike and car with some probability. So the label output vector might be [0,1,1] when only mototbikes and cars are present. And then for each 1 i.e. for motorbike and car we can output a probability vector indicating which make it is. 
%
\end{document}


