\documentclass{article}
\newcommand{\beq}{\begin{equation}}
\newcommand{\eeq}{\end{equation}}
\newcommand{\ber}{\begin{eqnarray}}
\newcommand{\eer}{\end{eqnarray}}
\newcommand{\nn}{\nonumber}
\newcommand{\dd}[2]{\frac{d}{d{#2}}{(#1)} }
\newcommand{\pdd}[2]{\frac{\partial{#1}}{\partial{#2}}}
\newcommand{\bw}{{\mathbf{w}}}
\newcommand{\bff}{\mathbf{f}}
\newcommand{\bPi}{\mathbf{\Pi}}
\newcommand{\bQ}{{\mathbf{Q}}}
\newcommand{\bq}{{\mathbf{q}}}
\newcommand{\bK}{{\mathbf{K}}}
\newcommand{\bU}{{\mathbf{U}}}
\newcommand{\bS}{{\mathbf{S}}}
\newcommand{\bV}{{\mathbf{V}}}

\usepackage{amsmath}
\usepackage{amsfonts}
\usepackage[hyphens]{url}
\usepackage{amssymb} 
\usepackage[utf8]{inputenc} 
%\usepackage[ngerman]{babel} 
\usepackage[T1]{fontenc}
\usepackage[margin=2.5cm]{geometry}
\usepackage{listings}
\usepackage{tikz}
\usepackage{cancel}
\definecolor {processblue}{cmyk}{0.96,0,0,0}
\usetikzlibrary {positioning}
\usepackage{hyperref}
\begin{document}
\title{Expectation}
\author{Nachiket Gokhale}
\date{\today}
\maketitle
\section{Expectation}
Let a fair die be tosses. The set out outcomes is ${1,2,3,4,5,6}$. $E(x)$ is $\frac{1}{6}(1+2+3+4+5+6)=3.5$. What is $E(xE(x))$. One way to understand it is to consider $xE(x)$ as another random variable capable of taking values ${3.5,7,10.5,14,17.5,21}$. So, $E(xE(x))=\frac{1}{6}(3.5+7+10.5+14+17.5+21)=3.5*\frac{1}{6}(1+2+3+4+5+6)=3.5^2=E(x)^2$.
\section{Total Expectation}
From \url{https://en.wikipedia.org/wiki/Law_of_total_expectation}.
\beq
E(X) = E(E(X|Y))
\eeq
Proof:
\ber
E(E(X|Y)) &=& E\Big[\sum_{x}xP(X=x|Y)\Big] \\
&=& \sum_y\Big[\sum_{x}xP(X=x|Y)\Big]P(Y=y) \\
&=& \sum_y\sum_xxP(X=x,Y=y) \\
&=& \sum_x x \sum_yP(X=x,Y=y) \text{ assuming sums can be interchanged } \\
&=& \sum_x x P(X=x) \\
&=& E(X)
\eer
Note: $P(X=y,Y=y)=P(X=x|Y=y)P(y)$ \text { because } $P(x\cup{y}) = P(x|y)p(y)$.
\end{document}

