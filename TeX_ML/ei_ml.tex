% Send to Mandar,Jayesh,Arnab,Deepak and perhaps Rahee
\documentclass{article}
\newcommand{\beq}{\begin{equation}}
\newcommand{\eeq}{\end{equation}}
\newcommand{\ber}{\begin{eqnarray}}
\newcommand{\eer}{\end{eqnarray}}
\newcommand{\nn}{\nonumber}
\newcommand{\dd}[2]{\frac{d}{d{#2}}{(#1)} }
\newcommand{\pdd}[2]{\frac{\partial}{\partial{#2}}{(#1)} }
\usepackage{amsmath}
\usepackage{amsfonts}
\usepackage{url}
\usepackage{graphicx}
\usepackage{caption}
\usepackage{subcaption}
\usepackage{multirow}
\begin{document}
\title{Stiffness parameter identification using Convolutional Neural Networks in TensorFlow}
\author{Nachiket Gokhale\footnote{Conversations with Paul Barbone, Arnab Majumdar and Mandar Kulkarni are acknowledged and much appreciated.}\\gokhalen@gmail.com}
\date{\today}
\maketitle
\section{Introduction}
\section{Binary classification}
% https://stackoverflow.com/questions/31324218/scikit-learn-how-to-obtain-true-positive-true-negative-false-positive-and-fal

%By definition a confusion matrix C is such that C[i, j] is equal to the number of observations known to be in group i but predicted to be in group j.

%Thus in binary classification, the count of true negatives is C[0,0], false negatives is C[1,0], true positives is C[1,1] and false positives is C[0,1].

% CM = confusion_matrix(y_true, y_pred)

% TN = CM[0][0]
% FN = CM[1][0]
% TP = CM[1][1]
% FP = CM[0][1]

\section{Things to do}
% Custom activation function
% https://stackoverflow.com/questions/43915482/how-do-you-create-a-custom-activation-function-with-keras
\begin{enumerate}
\item{Checkpoint and save best weights instead of just running for $512$ epochs. https://stackoverflow.com/questions/42666046/loading-a-trained-keras-model-and-continue-training}
\item{Also, checkpoint and restart}
\item{}
\item{Use prior knowledge to improve performance. e.g. if we know that radius can lie between 1 pixel and 3 pixels, perhaps we can use that information to improve performance. Similarly if we know that stiffness can lie between specifed ranges, then use that knowledge to constrain NN output}
\end{enumerate}
\end{document}
