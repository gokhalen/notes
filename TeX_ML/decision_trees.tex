\documentclass{article}
\newcommand{\beq}{\begin{equation}}
\newcommand{\eeq}{\end{equation}}
\newcommand{\ber}{\begin{eqnarray}}
\newcommand{\eer}{\end{eqnarray}}
\newcommand{\nn}{\nonumber}
\newcommand{\dd}[2]{\frac{d}{d{#2}}{(#1)} }
\newcommand{\ddeps}{\frac{d}{d\epsilon}\Big{|}_{\rightarrow{0}}}
\newcommand{\pdd}[2]{\frac{\partial{#1}}{\partial{#2}}}

\usepackage{graphicx}
\usepackage{amsmath}
\usepackage{amsfonts}
\usepackage{url}
\begin{document}
\title{Decision trees}
\author{Nachiket Gokhale}
\date{\today}
\maketitle
\section{Geron: Out of bag evaluation}
\url{https://stats.stackexchange.com/questions/562284/proof-out-of-bag-evaluation}
\textbf{Question} "In Geron's book "Hands-on Machine Learning with Scikit-Learn and Tensorflow" there is this sentence on page 187 "By default a BaggingClassifier samples m training instances with replacement (bootstrap=True), where m is the size of the training set. This means that only about 63\% of the training instances are sampled on average for each predictor."  And in the footnote he mentions this ratio approaches $1-\exp(-1)$ as $m$ grows (the author means that approaches $\infty$, I think). How should I prove this? I have no idea apart from: there are $m^m$ possible training sets."\\
\textbf{Answer} ``The probability of a particular sample in the training set not appearing in the bootstrap sample is 
$$p=\left(\frac{m-1}{m}\right)^m=\left(1-\frac{1}{m}\right)^m\rightarrow e^{-1}$$ as $m$ grows. That means probability of this particular sample being present in the bootstrap sample is $1-p=1-1/e\approx 0.63$. This means, on average, $63\%$ of the training set is expected to be in the bootstrapped sample.''

\textbf{Comment} ``This means that if we pick a sample of m points with replacement from m datapoints, on average, $e^{-1}$\% are not going to be picked''. This becomes that tree's out-of-bag
\end{document}
