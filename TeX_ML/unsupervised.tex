\documentclass{article}
\newcommand{\beq}{\begin{equation}}
\newcommand{\eeq}{\end{equation}}
\newcommand{\ber}{\begin{eqnarray}}
\newcommand{\eer}{\end{eqnarray}}
\newcommand{\nn}{\nonumber}
\newcommand{\dd}[2]{\frac{d}{d{#2}}{(#1)} }
\newcommand{\ddeps}{\frac{d}{d\epsilon}\Big{|}_{\rightarrow{0}}}
\newcommand{\pdd}[2]{\frac{\partial{#1}}{\partial{#2}}}

\usepackage{graphicx}
\usepackage{amsmath}
\usepackage{amsfonts}
\usepackage{url}
\begin{document}
\title{Unsupervised}
\author{Nachiket Gokhale}
\date{\today}
\maketitle
\section{K-means clustering}
1. Algorithm is guaranteed to converge but may not convert to the correct solution. To tackle this, we need an ensemble.\\
2 Scaling issues. Data MUST be scaled.\\
3. Algorithm stops when no centroids move\\
4. Goodness of clustering: Inertia (lower the intertia, better is clustering). But, you can use $k=n$ and inertia will be zero. (overfitting) \textbf{Understand Silhouette coeff}\\
5. Prof's comment in chat: But note that means is guaranteed to converge only for Euclidean distances

\section{DB-Scan}
1. DBScan figures out number of clusters for itself.


\section{Anamoly detection}
\textbf{Anamoly} we don't want. \textbf{Novelty} we want. Means and GMMs for the model are learned from the data.

\end{document}
